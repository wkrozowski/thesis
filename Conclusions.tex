\chapter{Conclusions and Future Work}
\label{chapterlabel4}
We conclude the thesis, by summarising the key contributions and sketching the potential directions for future work.
\section{Completeness theorems for behavioural distances}
The starting point of the first part of the thesis was a paper by Bacci, Bacci, Larsen and Mardare~\cite{Bacci:2018:Bisimilarity}, who used a (relaxed version of) quantitative equational theories~\cite{Mardare:2016:Quantitative} to axiomatise probabilistic bisimilarity distance between terms of probabilistic process algebra of Stark and Smolka~\cite{Stark:2000:Complete}. While that result heavily hinged on properties of Kantorovich lifting used to define the behavioural distance, the key observation was that properties necessary for completeness proof are not exclusive to Kantorovich lifting, but rather can be adapted to other instances of behavioural distances stemming from the abstract coalgebraic framework~\cite{Baldan:2018:Coalgebraic}. 

In \Cref{chapter2}, we have focused on the simplest and most intuitive instantiation of the coalgebraic framework in the case of deterministic automata. As a central contribution, we have obtained a sound and complete axiomatisation of the shortest-distinguishing-word distance between regular expressions. An interesting difference between our result and previous work was the fact that Kleene's star does not break non-expansivity (unlike $\mu$-recursion operator in the case of probabilistic bisimilarity distance~\cite{Bacci:2018:Bisimilarity}) that allowed us to rely on the framework of quantitative equational theories, without any ad-hoc modifications to it.

In \Cref{chapter3}, we looked at a more involved case of Milner's charts~\cite{Milner:1984:Complete}, a straightforward generalisation of nondeterministic automata with variable outputs that presents a compelling setting to study behavioural distances, as it shifts focus from linear-time behaviours to branching-time semantics and represents a crucial step towards more complicated models, such as weighted transition systems~\cite{Larsen:2011:Metrics}. Rather than directly following Milner and using an involved process algebraic syntax with binders and $\mu$-recursion operator, we have relied on a compositional, string diagrammatic syntax, building on a previous line of work on string diagrammatic approaches to automata theory~\cite{Piedeleu:2024:Complete,antoinecsl2025}. 

One of the key contributions of~\Cref{chapter3} was providing an axiomatic system for reasoning about distances between string diagrams. Besides the recent work of Lobbia et al~\cite{Lobbia:2024:Quantitative}, who provided basic examples of total variation distance between stochastic matrices and preorders on matrices, our work is the first one to propose a sound and complete quantitative calculus of string diagrams. Despite multiple similarities, our axiomatisation cannot be expressed in the framework of Lobbia et al~\cite{Lobbia:2024:Quantitative}, which only permits purely equational axioms. Reconciling these two, by providing a more general framework for quantitive axiomatisations of string diagrams permitting implicational rules and axiom schemes is an interesting research direction.

As much as the usage of Hausdorff distance in defining behavioural distance of charts led to a more involved completeness proof, the proof strategy was essentially the same as in \Cref{chapter2} and in the work of Bacci, Bacci, Larsen and Mardare~\cite{Bacci:2018:Bisimilarity}. This suggests the possibility of developing a more generic framework of axiomatisations of behavioural distances parametric on the branching type of the system and the associated lifting. One of the directions could be following the work of Schmid et al~\cite{Schmid:2022:Processes}, who generalised Milner's charts and an algebra of regular behaviours to coalgebras for the type functor $T(V + \alphabet \times (-))$, where $T \colon \Set \to \Set$ is an underlying functor of a monad $\monadT$ and a family of process algebras parametric on algebraic operations appearing in the presentation of $\monadT$. We envision that such a framework would rely on liftings of $T \colon \Set \to \Set$ to $\overline{T} : \PMet \to \PMet$ that are nonexpansive with respect to sup norm and the lifted monad $\overline{\monadT}$ can be presented as a quantitative theory in which one can arbitrarily closely approximate the distance between terms, from the approximations of distances between the variables. A good starting point could be the class of quantitative theories studied by Mardare et al~\cite{Mardare:2016:Quantitative}, where all the axioms are so-called \emph{continuous equation schematas}, that precisely enable such approximations. Additionally, it would be interesting to see if such a general framework of axiomatisations of behavioural distances could be reconciled with work on fixpoint extension of quantitative equational theories~\cite{Mardare:2021:Fixed}. 
\section{Probabilistic language equivalence}
The second part of the thesis explored obtaining a sound and complete axiomatisation of language equivalence of generative probabilistic transition systems~\cite{Glabbeek:1995:Reactive} through the syntax of probabilistic regular expressions generalising Kleene's regular expressions to a probabilistic setting. Our starting point were recent hard results on properties of convex algebras and fixpoints that Milius~\cite{Milius:2018:Proper}, Sokolova and Woracek~\cite{Sokolova:2015:Congruences,Sokolova:2018:Proper}. Those results enabled the use of an abstract framework of proper functors~\cite{Milius:2018:Proper} that allowed us to reduce an involved completeness problem to a generalisation of automata-theoretic results that were studied by Salomaa~\cite{Salomaa:1966:Two}, Kleene~\cite{Kleene:1951:Representation} and Brzozowski~\cite{Brzozowski:1964:Expressions} more than fifty years ago. Our work provided further evidence that proper functors are a good abstraction for coalgebraic completeness theorems, that cast completeness as proving a certain universal property obtained by uniquely solving finite systems of fixpoint equations. 

In the conclusion of~\Cref{chapter4}, we left several interesting areas for future work, such as obtaining an algebraic axiomatisation in the style of Kozen~\cite{Kozen:1994:Completeness}. Besides this concrete direction, a natural broader research direction would be to extend a generic calculus for weighted automata of Bonsangue et al~\cite{Bonsangue:2013:Sound} to coalgebras over arbitrary proper functors. We stipulate that this would crucially rely on extracting the syntax from the presentation of the functor, as was the case in the work of Bonsangue and Kurz~\cite{Bonsangue:2006:Presenting}, Silva~\cite{Silva:2010:Kleene} and more generally in coalgebraic modal logic~\cite{Schroder:2008:Expressivity}.

Finally, there is a natural research question intersecting both parts of the thesis, namely axiomatising behavioural distance between probabilistic languages denoted by probabilistic regular expressions. The generic framework of coalgebraic behavioural distances can be applied to linear-time behaviours obtained via generalised determinisation~\cite{Silva:2010:Generalizing} through lifting distributive laws~\cite{Baldan:2018:Coalgebraic}. In the case of generative probabilistic transition systems, this would yield a variant of a total variation distance between observable words. An immediate obstacle is that one of the key properties used in quantitative completeness theorems in the first part of the thesis was the finiteness of the state-spaces. Unfortunately, as mentioned in~\Cref{chapter4}, determinising a finite-state generative probabilistic transition system always results in an infinite state-space. The usage of proper functors was crucial in allowing us to reduce completeness to looking at state-spaces that freely generated convex combinations of finitely many elements. We hope that a similar approach could be helpful in a quantitative case.