\newcommand{\printpublication}[1]{\AtNextCite{\defcounter{maxnames}{99}}\fullcite{#1}}

% theorem environment setup
\newtheorem{theorem}{Theorem}[section]
\newtheorem{lemma}[theorem]{Lemma}
\newtheorem{corollary}[theorem]{Corollary}
\newtheorem{proposition}[theorem]{Proposition}
\newtheorem*{proposition*}{Proposition}

\theoremstyle{definition}
\newtheorem{definition}[theorem]{Definition}
\newtheorem{question}{Question}
\newtheorem*{question*}{Question}
\newtheorem{assumption}{Assumption}

\theoremstyle{remark}
\newtheorem{remark}[theorem]{Remark}
\newtheorem{example}[theorem]{Example}

% Things shared between chapters
\newcommand{\alphabet}{{A}}
\newcommand{\seq}{\mathbin{;}} %Sequential composition of (probabilistic) regular expressions
\newcommand{\zero}{\mathsf{0}} %0 of a (probabilistic) regular expression
\newcommand{\one}{\mathsf{1}} %1 of a (probabilistic) regular expression
\newcommand{\sem}[1]{\left\llbracket{#1}\right\rrbracket}
\newcommand{\cat}[1]{\mathcal{#1}} %Category
\newcommand*\circlednum[1]{\tikz[baseline=(char.base)]{
            \node[shape=circle,draw,inner sep=1pt] (char) {#1};}} %number in the circle

%number sets
\newcommand{\N}{\mathbb{N}}
\newcommand{\Z}{\mathbb{Z}}
\newcommand{\R}{\mathbb{R}}
\newcommand{\Q}{\mathbb{Q}}

%Some generic categories
\def\A{\cat{A}}
\def\C{\cat{C}}
\def\D{\cat{D}}
\def\J{\cat{J}}


\newcommand{\funF}{\mathcal{F}} % Arbitrary functor

\newcommand{\Obj}{\mathsf{Obj}} % Objects of category
\newcommand{\forget}{\mathcal{U}}

\def\subto{\hookrightarrow}
\def\monoto{\rightarrowtail}
\def\epito{\twoheadrightarrow}
\def\ot{\leftarrow}


\def\Set{{\mathsf{Set}}}
\def\Rel{{\mathsf{Rel}}}

\def\pset{\mathcal{P}}

% generic coalgebra
\newcommand{\beh}[1]{\mathsf{beh}_{#1}} 

\def\coa{\mathop{\mathsf{Coalg}}}
\def\alg{\mathop{\mathsf{Alg}}}
\def\coaf{\mathop{\mathsf{Coalg}_{\mathsf{fp}}}}
\def\coalf{\mathop{\mathsf{Coalg}_{\mathsf{lf}}}}
\def\coafg{\mathop{\mathsf{Coalg}_{\mathsf{fg}}}}
\def\coalfp{\mathop{\mathsf{Coalg}_{\mathsf{lfp}}}}
\def\coafr{\mathop{\mathsf{Coalg}_{\mathsf{free}}}}
\def\beheq{\sim_{\mathsf{b}}} %behavioural equivalence

\newcommand{\gen}[2]{\langle #1 \rangle_{#2}}

% common behavioural distances stuff
\newcommand{\disteq}[1]{\equiv_{#1}} % rational-indexed equality
\newcommand{\e}{\varepsilon} % epsilon for distances
\newcommand{\PMet}{\mathsf{PMet}}
\newcommand{\bd}{\mathsf{bd}}

%fixpoints of functors
\def\rho{\varrho}
\def\theta{\vartheta}
\def\phi{\varphi}

% chapter two only
\newcommand{\RExp}{\mathsf{RExp}}

\newcommand{\objr}{\blacktriangleright}
\newcommand{\objl}{\blacktriangleleft}
\newcommand{\arrowright}{\tikzfig{lr-order}}
\newcommand{\arrowleft}{\tikzfig{rl-order}}
\newcommand{\emptyword}{\mathtt{e}}
\newcommand{\acirel}{{\mathrel{\dot\equiv}}}
\newcommand{\aciq}{{\RExp}/{\acirel}}
\newcommand{\eR}{\overline{\mathbb{R}}} %Extended reals
\newcommand{\funDFA}{\mathcal{H}}
\usepackage{xifthen}

\providecommand*{\ifempty}[3]{\ifthenelse{\isempty{#1}}{#2}{#3}}
\newcommand{\set}[2]{\left\{ #1 \ifempty{#2}{}{\mid #2} \right\}}
%% Terms and Deduction Systems
\newcommand{\TT}[2][]{\mathbb{T}\ifempty{#2}{\ifempty{#1}{}{(#1)}}{\ifempty{#1}{(#2)}{(#1,#2)}}}  %% Set of terms
\newcommand{\Sub}[1][]{\ifempty{#1}{\mathcal{S}}{\mathcal{S}(#1)}} %% Set of substitutions
\newcommand{\E}[1][]{\ifempty{#1}{\mathcal{E}}{\mathcal{E}(#1)}} %% Set of quantitative equations
\newcommand{\U}[1][]{{\mathcal{U}^{#1}}} %% Theory induced by a quantitative deduction system
\newcommand{\KK}[2][]{\mathbb{K}\ifempty{#1}{(#2)}{(#1,#2)}} %% Class of models for theory
\newcommand{\freemodel}[1][\vdash]{{\mathcal{T}_{#1}}} %% Freely-generated model

\newcommand{\Refl}{\textsf{Refl}} %% Reflexivity Axiom
\newcommand{\Symm}{\textsf{Symm}} %% Symmetry Axiom
\newcommand{\Triang}{\textsf{Triang}} %% Triangular inequality Axiom
\newcommand{\Max}{\textsf{Max}} %% Max Axiom
\newcommand{\Cont}{\textsf{Cont}} %% Archimedian Axiom
\newcommand{\Nexp}{\textsf{NExp}} %% Non-expansivity Axiom
\newcommand{\Subst}{\textsf{Subst}} %% Substitution Axiom
\newcommand{\Cut}{\textsf{Cut}} %% Cut rule
\newcommand{\Assum}{\textsf{Assum}} %% Assumption rule
\newcommand{\Bone}{\textsf{B1}}
\newcommand{\Btwo}{\textsf{B2}}
\newcommand{\SC}{\textsf{SC}}
\newcommand{\SA}{\textsf{SA}}
\newcommand{\Alpha}{\textsf{Alpha}}
\newcommand{\Unfold}{\textsf{Unfold}}
\newcommand{\Unguard}{\textsf{Unguard}}
\newcommand{\Fix}{\textsf{Fix}}
\newcommand{\Cong}{\textsf{Cong}}
\newcommand{\Top}{\textsf{Top}}
\newcommand{\Conv}{\textsf{IB}}
\newcommand{\Dist}{\textsf{Dist}}
\newcommand{\Pref}{\textsf{Pref}}
\newcommand{\Entr}{\textsf{Entr}}
\newcommand{\Distr}{\textsf{Distr}}
\newcommand{\dPref}[1][\lambda]{\textsf{$#1$-Pref}}
\newcommand{\qalgA}{\mathcal{A}} %% Generic quantitative algebra
\newcommand{\qalgB}{\mathcal{B}} %% 2nd generic quantitative algebra
\newcommand{\langmetric}{d_{\pset(\alphabet^{\ast})}}

%%%%%%%%%%%%%%%%%%%%%%%%%%%%%%%%%%%%%%%%%%%%%%%%%%%%%%%%%%%%%%%%%%%%%%%%%%%%%
%                          ROBIN'S TIKZ STUFF
%%%%%%%%%%%%%%%%%%%%%%%%%%%%%%%%%%%%%%%%%%%%%%%%%%%%%%%%%%%%%%%%%%%%%%%%%%%%%

\pgfdeclarelayer{edgelayer}
\pgfdeclarelayer{nodelayer}
\pgfsetlayers{edgelayer,nodelayer,main}

\definecolor{light-gray}{gray}{.5}
\tikzstyle{right arrow} = [->]
\tikzstyle{none}=[inner sep=0pt]
\tikzstyle{plain}=[inner sep=0pt]
\tikzstyle{black}=[circle, draw=black, fill=black, inner sep=0pt, minimum size=4pt]
\tikzstyle{black-faded}=[circle, draw=light-gray, fill=light-gray, inner sep=0pt, minimum size=4pt]
\tikzstyle{white}=[circle, draw=black, fill=white, inner sep=0pt, minimum size=4.5pt]
\tikzstyle{white-faded}=[circle, draw=light-gray, fill=white, inner sep=0pt, minimum size=4.5pt]
\tikzstyle{delay}=[fill=black, regular polygon, regular polygon sides=3,rotate=-90, scale=.55]
\tikzstyle{delay-op}=[fill=black, regular polygon, regular polygon sides=3,rotate=90, scale=.55]
\tikzstyle{reg}=[draw, fill=white, rounded rectangle, rounded rectangle left arc=none, minimum height=1.2em, minimum width=1.4em, node font={\scriptsize}]
\tikzstyle{coreg}=[draw, fill=white, rounded rectangle, rounded rectangle right arc=none, minimum height=1.2em, minimum width=1.4em, node font={\scriptsize}]
\tikzstyle{basicb}=[draw, fill=white, rectangle, rounded corners, minimum height=1.6em, minimum width=1.4em]
\tikzstyle{basic box}=[draw, fill=white, rectangle, rounded corners, minimum height=1.6em, minimum width=1.4em]
\tikzstyle{smallb}=[draw, fill=white, rectangle, rounded corners, minimum height=1.2em, minimum width=1.4em, node font={\scriptsize}]
\tikzstyle{rcoreg}=[draw=red, fill=white, rounded rectangle, rounded rectangle right arc=none, minimum height=1.2em, minimum width=1.4em, node font={\scriptsize}]
\tikzstyle{regb}=[draw, fill=black, rounded rectangle, rounded rectangle left arc=none, minimum height=1.2em, minimum width=1.4em, node font={\scriptsize}]
\tikzstyle{regbw}=[draw, left color=black, right color=white, middle color=white, rounded rectangle, rounded rectangle left arc=none, minimum height=1.2em, minimum width=1.4em, node font={\scriptsize}]
\tikzstyle{regwb}=[draw, left color=white, right color=black, middle color=white, rounded rectangle, rounded rectangle left arc=none, minimum height=1.2em, minimum width=1.4em, node font={\scriptsize}]
\tikzstyle{coregb}=[draw, fill=black, rounded rectangle, rounded rectangle right arc=none, minimum height=1.2em, minimum width=1.4em, node font={\scriptsize}]
\tikzstyle{coregbw}=[draw, left color=black, right color=white, middle color=white, rounded rectangle, rounded rectangle right arc=none, minimum height=1.2em, minimum width=1.4em, node font={\scriptsize}]
\tikzstyle{coregwb}=[draw, left color=white, right color=black, middle color=white, rounded rectangle, rounded rectangle right arc=none, minimum height=1.2em, minimum width=1.4em, node font={\scriptsize}]
\tikzstyle{rn}=[circle, draw=red, fill=red, inner sep=0pt, minimum size=4pt]
\tikzstyle{wrn}=[circle, draw=red, fill=white, inner sep=0pt, minimum size=4pt]
\tikzstyle{place}=[circle, draw=black, fill=white, inner sep=0pt, minimum size=9pt]
\tikzstyle{act}=[circle, draw=black, fill=white, inner sep=0pt, minimum size=4.5pt]
\tikzstyle{coact}=[draw, fill=white, rounded rectangle, rounded rectangle right arc=none, minimum height=.7em, minimum width=.9em, node font={\scriptsize}]
\tikzstyle{basic rounded box}=[draw, fill=white, rectangle, rounded corners, minimum height=1.6em, minimum width=1.4em]
\tikzstyle{small rounded box}=[draw, fill=white, rectangle, rounded corners, minimum height=1.2em, minimum width=1.4em, node font={\scriptsize}]
\tikzset{
BWmatrix/.pic={
    \coordinate (center) at (0,0);
    \filldraw[fill=white, draw=black, line width=1pt] (.5,0) 
        [rounded corners=14pt] -- (1,0) 
        [rounded corners=14pt] -- (1,1)
        [rounded corners=0pt] -- (.5,1) 
        [rounded corners=0pt] -- cycle;
    \filldraw[fill=black, draw=black, line width=1pt] (0,0) 
        -- (.5,0) 
        -- (.5,1)
        -- (0,1) 
        -- cycle;
   },
pics/BWmatrix/.default=0.2
}


% Petri net styles
\tikzstyle{pl}=[circle,thick,draw=black!75,fill=white,minimum size=17pt]
\tikzstyle{port}=[circle, fill,inner sep=1.2pt]
\tikzstyle{transition}=[rectangle,thick,draw=black!75,
  			  fill=black!20,minimum size=7pt]

\tikzstyle{arrow}=[->]

\newcommand{\tikzfig}[1]{
\tikzset{x=1em, y=2.1ex}
\InputIfFileExists{#1.tikz}{}{\input{./tikz/#1.tikz}}
\tikzset{x=1em, y=1.5ex}
}

\newcommand{\ctikzfig}[1]{%
\begin{center}
  \tikzfig{#1}
\end{center}}


\newcommand{\eqtikzfig}[1]{%
\begin{equation}
  \tikzfig{#1}
\end{equation}}

% Directed box
\newcommand{\dbox}[3]{
\begin{tikzpicture}
	\begin{pgfonlayer}{nodelayer}
		\node [style=none] (6) at (1.5, 0) {};
		\node [style=basicb] (10) at (0, 0) {$#1$};
		\node [style=none] (11) at (-1.25, 0) {};
		\node [style=none] (13) at (-2, 0) {};
		\node [style=none] (14) at (2, 0) {};
		\node [style=none] (15) at (1.75, 0.75) {\scriptsize $#3$};
		\node [style=none] (16) at (-1.75, 0.75) {\scriptsize $#2$};
	\end{pgfonlayer}
	\begin{pgfonlayer}{edgelayer}
		\draw [->] (13.center) to (11.center);
		\draw (6.center) to (14.center);
		\draw (11.center) to (10);
		\draw [->] (10) to (6.center);
	\end{pgfonlayer}
\end{tikzpicture}
}


% Trace with actions
\newcommand{\traceaction}[5]{
\begin{tikzpicture}
	\begin{pgfonlayer}{nodelayer}
		\node [style=none] (0) at (-0.75, 1.25) {};
		\node [style=none] (1) at (0.75, 1) {};
		\node [style=none] (2) at (-0.25, 1.75) {};
		\node [style=none] (3) at (-0.75, -0.25) {};
		\node [style=none] (4) at (0.75, -0.25) {};
		\node [style=none] (5) at (-0.25, -0.75) {};
		\node [style=none] (6) at (-0.75, 1) {};
		\node [style=none] (7) at (0.75, 1.25) {};
		\node [style=none] (8) at (2.5, 0) {};
		\node [style=none] (9) at (0.75, 0) {};
		\node [style=none] (10) at (0.25, -0.75) {};
		\node [style=none] (11) at (0.25, 1.75) {};
		\node [style=none] (12) at (0, 0.5) {$#1$};
		\node [style=none] (13) at (4, 0.5) {\scriptsize $#3$};
		\node [style=none] (14) at (2.5, 2.5) {};
		\node [style=none] (15) at (-1.5, 2.5) {};
		\node [style=none] (16) at (-2.75, 0.5) {\scriptsize $#2$};
		\node [style=none] (17) at (-1.75, 0) {};
		\node [style=none] (18) at (-0.75, 0) {};
		\node [style=none] (20) at (2.5, 1) {};
		\node [style=none] (21) at (3.5, 2.75) {\scriptsize $#4$};
		\node [style=none] (22) at (-1.5, 1) {};
		\node [style=reg] (23) at (1.75, 1) {$#5$};
		\node [style=none] (24) at (4.25, 0) {};
		\node [style=none] (25) at (-3, 0) {};
	\end{pgfonlayer}
	\begin{pgfonlayer}{edgelayer}
		\draw [semithick, in=0, out=-90] (4.center) to (10.center);
		\draw [semithick, in=-90, out=180] (5.center) to (3.center);
		\draw [semithick, in=180, out=90] (0.center) to (2.center);
		\draw [semithick, in=90, out=0] (11.center) to (7.center);
		\draw [semithick] (2.center) to (11.center);
		\draw [semithick] (7.center) to (4.center);
		\draw [semithick] (10.center) to (5.center);
		\draw [semithick] (3.center) to (0.center);
		\draw (15.center) to (14.center);
		\draw (6.center) to (22.center);
		\draw (1.center) to (20.center);
		\draw [->, bend right=90, looseness=2.25] (20.center) to (14.center);
		\draw [->, bend right=90, looseness=2.25] (15.center) to (22.center);
		\draw (8.center) to (24.center);
		\draw [->] (9.center) to (8.center);
		\draw (17.center) to (18.center);
		\draw [->] (25.center) to (17.center);
	\end{pgfonlayer}
\end{tikzpicture}}


\newcommand{\myeq}[1]{\mathrel{\overset{\makebox[0pt]{\mbox{\normalfont\tiny\sffamily (#1)}}}{=}}}
\newcommand{\myleq}[1]{\mathrel{\overset{\makebox[0pt]{\mbox{\normalfont\tiny\sffamily (#1)}}}{\leq}}}
\newcommand{\mygeq}[1]{\mathrel{\overset{\makebox[0pt]{\mbox{\normalfont\tiny\sffamily (#1)}}}{\geq}}}
%%%%%%%%%%%%%%%%%%%%%%%%%%%%%%%%%%%%%%%%%%%%%%%%%%%%%%%%%%%%%%%%%%%%%%%%%%%%%%%%

% chapter three only
\newcommand{\idright}{\tikzfig{lr-order}}
\newcommand\scalar[1]{
  \tikz {
    \node[ha] (ha) {$#1$};
    \draw (ha.west) -- ++(-0.75, 0);
    \draw (ha.east) -- ++(0.75, 0);
  }
}

% chapter four only
\newcommand{\PExp}{\mathsf{PExp}}
\def\pca{{\mathsf{PCA}}}
\def\smod{\S\textsf{-Mod}}