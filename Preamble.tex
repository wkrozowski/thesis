% I may change the way this is done in a future version, 
%  but given that some people needed it, if you need a different degree title 
%  (e.g. Master of Science, Master in Science, Master of Arts, etc)
%  uncomment the following 3 lines and set as appropriate (this *has* to be before \maketitle)
% \makeatletter
% \renewcommand {\@degree@string} {Master of Things}
% \makeatother

\title{Completeness Theorems for Behavioural Metrics and Equivalences}
\author{Wojciech Krzysztof R\'{o}\.{z}owski}
\department{Department of Computer Science}

\maketitle
\makedeclaration

\begin{abstract} % 300 word limit
My research is about stuff.

It begins with a study of some stuff, and then some other stuff and things.

There is a 300-word limit on your abstract.
\end{abstract}

\begin{impactstatement}
\textbf{Outside academia:} The discipline of formal verification enables making precise logical statements about computer systems to guarantee their correctness. This relies on the study of models of computation, which are mathematical objects representing the semantics of systems of interest. In theoretical computer science, it is customary to model computations as transition systems. This thesis is part of a larger research programme aimed at providing specification languages. for representing transition systems and studying formal systems for reasoning about the equivalence or similarity of systems represented by the syntax. One of the central kind of problems attached to this field of research are completeness problems, that concern showing that every semantic equivalence or similarity of transition systems can be witnessed through the means of axiomatic manipulation. Having a complete axiomatisation allows one to fully resort to syntactic reasoning, which is particularly amenable to implementation and thus desirable from the automated reasoning point of view. 

Kleene Algebra (KA)~\cite{Kozen:1994:Completeness} is a central example of such specification language. KA is a foundation of tools relevant to industry, such as NetKAT~\cite{Anderson:2014:NetKAT}, which enables reasoning about the behaviour of packet-passing Software-Defined Networks, or cf-GKAT~\cite{Zhang:2025:CFGKAT}, which can be used to certify the correctness of decompilation algorithms. This thesis particularly focuses on the semantic notions of behavioural distances and probabilistic language equivalence, where it makes its main contributions. Behavioural distances~\cite{Breugel:2001:Towards,Desharnais:2004:Metrics} replace the strict notion of equivalence of states with a more liberal quantitative measure of dissimilarity. This is particularly desirable for stochastic or probabilistic systems, where a tiny observed perturbation would lead to inequivalence of states. Behavioural distances have been successfully applied to Markov Decision Processes within Reinforcement Learning (RL), with the key example being MICo (Matching under Independent Couplings) distance~\cite{Castro:2021:MICo}. Additionally, probabilistic language equivalence, axiomatised in Chapter 4 of this thesis, underpins Apex~\cite{Kiefer:2012:APEX}, an automated equivalence checker for probabilistic programs.

\textbf{Inside academia:} This thesis makes original contributions within the field of theoretical computer science. The results in \Cref{chapter2} are an adaptation of a concrete completeness result for behavioural distance of probabilistic transition systems to an instance of an abstract coalgebraic framework. Besides the basic examples provided recently by Lobbia et al~\cite{Lobbia:2024:Quantitative}, the content of \Cref{chapter3} is the first work to propose a complete axiomatisation of a quantitative calculus of string diagrams through a systematic axiomatic foundation. Finally, \Cref{chapter4} provides an alternative axiomatisation of language equivalence of generative probabilistic transition systems~\cite{Glabbeek:1995:Reactive} through a simple generalisation of Kleene’s regular expressions. The completeness result makes use of recently developed theory of proper functors~\cite{Milius:2018:Proper} and provides further evidence that the use of coalgebras for proper functors provides a good abstraction for completeness theorems.

The entire material presented in this thesis has been accepted or is under review for several highly-ranked conferences in theoretical computer science. The results of this thesis have been disseminated to the computer science community through a series of seminar talks across the United Kingdom, the United States, and Germany. 
\end{impactstatement}

\begin{acknowledgements}
Acknowledge all the things!
\end{acknowledgements}

\setcounter{tocdepth}{2} 
% Setting this higher means you get contents entries for
%  more minor section headers.

\tableofcontents
\listoffigures
\listoftables

