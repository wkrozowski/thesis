\phantomsection
\addcontentsline{toc}{chapter}{Appendices}

% The \appendix command resets the chapter counter, and changes the chapter numbering scheme to capital letters.
%\chapter{Appendices}
\appendix
\chapter{Soundness of axioms for bisimilarity of probabilistic regular expressions}
\label{appendixlabel1}
Below, we recall the notions surrounding couplings of (sub)probability distributions, which will be used in one of the intermediate steps of the proof of soundness.
\begin{definition}[{\cite[Definition~2.1.2]{Hsu:2017:Probabilistic}}]\label{def:coupling}
    Given two subdistributions $\nu_1, \nu_2$ over $X$ and $Y$ respectively, a subdistribution $\nu$ over $X \times Y$ is called coupling if:
    \begin{enumerate}
        \item For all $x \in X$, $\nu_1(x) = \nu[\{x\}\times Y]$
        \item For all $y \in Y$, $\nu_2(y)= \nu[X \times \{y\}]$
    \end{enumerate} 
\end{definition}
It can be straightforwardly observed that a coupling $\nu$ of $(\nu_1, \nu_2)$ is finitely supported if and only if both $\nu_1$ and $\nu_2$ are finitely supported.
\begin{definition}[{\cite[Definition~2.1.7]{Hsu:2017:Probabilistic}}]\label{def:lifting}
    Let $\nu_1, \nu_2$ be subdistributions over $X$ and $Y$ respectively and let $R \subseteq X \times Y$ be a relation. A subdistribution $\nu$ over $X \times Y$ is a \emph{witness} for the $R$-\emph{lifting} of $(\nu_1, \nu_2)$ if:
    \begin{enumerate}
        \item $\nu$ is a coupling for $(\nu_1, \nu_2)$
        \item $\supp(\mu) \subseteq R$
    \end{enumerate}
\end{definition}
If there exists $\nu$ satisfying these two conditions, we say $\nu_1$ and $\nu_2$ are related by the \emph{lifting} of $R$ and write $\nu_1 \equiv_R \nu_2$. It can be immediately observed that $\nu_1 \equiv_R \nu_2$ implies $\nu_2 \equiv_{R^{-1}} \nu_1$.

Given a relation ${R}\subseteq{X \times Y}$ and a set $B \subseteq X$, we will write $R(B) \subseteq Y$ for the set given by $R(B)= \{y \in Y \mid (x,y) \in R \text{ and } x \in B\}$. We will write $R^{-1} \subseteq Y \times X$ for the converse relation given by $R^{-1} = \{(y,x) \in Y \times X \mid (x,y) \in R\}$. 
\begin{theorem}[{\cite[Theorem 2.1.11]{Hsu:2017:Probabilistic}}]\label{thm:coupling_theorem}
Let $\nu_1$, $\nu_2$ be subdistributions over $X$ and $Y$ respectively and let $R \subseteq X \times Y$ be a relation. Then the lifting $\nu_1 \equiv_R \nu_2$ implies $\nu_1[B] \leq \nu_2[R(B)]$ for every subset $B \subseteq X$. The converse holds if $\nu_1$ and $\nu_2$ have equal weight.
\end{theorem}
\begin{lemma}\label{lem:coupling_lemma}
Let $\nu_1$, $\nu_2$ be subdistributions over $X$ and $Y$ respectively and let $R \subseteq X \times Y$ be a relation. $\nu_1 \equiv_R \nu_2$ if and only if:
\begin{enumerate}
	\item For all $B \subseteq X$,  $\nu_1[B] \leq \nu_2[R(B)]$
	\item For all $C \subseteq Y$, $\nu_2[C] \leq \nu_1[R^{-1}(C)]$
\end{enumerate}
\end{lemma}
\begin{proof}
	Assume that $\nu_1 \equiv_R \nu_2$. Recall, that in such a case $\nu_2 \equiv_{R^{-1}} \nu_2$. Applying \cref{thm:coupling_theorem} yields (1) and (2) respectively.
	
	For the converse, assume (1) and (2) do hold. We have the following:
\begin{align*}
	|\nu_1| &= \nu_1[X] \\
	&\leq \nu_2[R(X)] \tag{1} \\
	&\leq \nu_2[Y] \tag{$R(X) \subseteq Y$}\\
	&= |\nu_2|
\end{align*}
By a symmetric reasoning involving (2), we can show that $|\nu_2| \leq |\nu_1|$ and therefore $|\nu_1| = |\nu_2|$. Since condition (1) holds, we can use \Cref{thm:coupling_theorem} to conclude that $\nu_1 \equiv_R \nu_2$.
\end{proof}


%\chapter{Another Appendix About Things}
%\label{appendixlabel2}
%(things)

%\chapter{Colophon}
%\label{appendixlabel3}
%\textit{This is a description of the tools you used to make your thesis. It helps people make future documents, reminds you, and looks good.}
%
%\textit{(example)} This document was set in the Times Roman typeface using \LaTeX\ and Bib\TeX , composed with a text editor. 
 % description of document, e.g. type faces, TeX used, TeXmaker, packages and things used for figures. Like a computational details section.
% e.g. http://tex.stackexchange.com/questions/63468/what-is-best-way-to-mention-that-a-document-has-been-typeset-with-tex#63503

% Side note:
%http://tex.stackexchange.com/questions/1319/showcase-of-beautiful-typography-done-in-tex-friends
