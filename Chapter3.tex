\chapter{A Diagrammatic Approach to Behavioural Distance of Nondeterministic Processes}
\label{chapter3}

\section{Preliminaries}\label{c3:sec:preliminaries}
\subsection{Charts}
Fix a set $V=\{v_1, v_2, \dots\}$ of \emph{variables} and $\Sigma$ of \emph{letters} respectively. A prechart is a triple $(Q,E,D)$, where $Q$ is a set of states, $D \subseteq Q \times \Sigma \times Q$ a finite labelled transition relation and $E \subseteq Q \times V$ is a finite output relation. Precharts can be thought as a generalisation of nondeterministic automata, where instead of acceptance, we deal with the notion of outputs. Moreover, when $D$ and $E$ are clear from the context, we will write $q \tr{a} q' \iff (q,a,q') \in D$ and $q \rhd v \iff (q,v) \in E$. A chart $C$ is a quadruple $(Q, s, D, E)$, where $(Q,D,E)$ is a prechart and $s \in Q$ is a distinguished start node. We call a chart finite if $Q$ is finite.
\begin{definition}[Strong Bisimulation]
	Let $C_i = (Q_i,D_i,E_i)$, $i \in \{1,2\}$ be precharts. A bisimulation between $C_1$ and $C_2$ is a relation ${R} \subseteq Q_1 \times Q_2$, such that \circlednum{1} if $(q_1,q_2)\in R$, then $E(q_1)=E(q_2)$, \circlednum{2} if $(q_1,q_2) \in R$ and $q_1 \tr{a} q'_1$, then there exists $q'_2 \in Q_2$, such that $q_2 \tr{a} q'_2$ and $(q'_1, q'_2) \in R$ and symmetrically. If $C_1$ and $C_2$ are charts, we say that they are bisimilar (denoted $C_1 \sim C_2$) if there exists a bisimulation between their underlying precharts that relates their start nodes.
\end{definition}
Using the above definition, we can also define the following:
\begin{definition}[Prechart homomorphism]
	 Let $C_i = (Q_i,D_i,E_i)$, $i \in \{1,2\}$ be precharts. We call a function $f \colon Q_1 \to Q_2$ a prechart homomorphism if the graph of $f$, given by $G(f) = \{(q,f(q)) \mid q \in Q_1\}$ is a bisimulation between $C_1$ and $C_2$.
\end{definition}
In other words, prechart homomorphisms preserve and reflect transitions. Given a chart $C=(Q,s,D,E)$ we say that a variable $v \in V$ is \emph{live} in $C$ if there exists a path of transitions $s \tr{a_1} \dots \tr{a_n} s' \rhd v$ or call it \emph{dead} otherwise. It can be easily observed that bisimulations and homomorphisms preserve the liveness of variables.

Later in this chapter, we will characterise the behavioural distance of precharts via Hennessy and Milner's \emph{stratification of bisimilarity}~\cite{hennessy:1985:algebraic} defined by the following:
	\begin{definition}[Stratification of bisimilarity]
	Let $C_i = (Q_i, D_i, E_i)$ for $i \in \{1,2\}$ be precharts. We can define a family $\{\sim^{(i)} \}_{i \in \N}$ of equivalence relations on $Q_1 \times Q_2$ given by the following. For all $(q_1,q_2) \in Q_1 \times Q_2$, we have that $q_1 \sim^{(0)} q_2$. Given $(q_1, q_2) \in Q_1 \times Q_2$, we have that $q_1 \sim ^{(n+1)} q_2$ if \circlednum{1} $E_1(q_1) = E_2(q_2)$, \circlednum{2} $q_1 \tr{a}_{C_1} q'_1$ implies that there exists $q'_2 \in Q_2$, such that $q_2 \tr{a} q'_2$ and $q'_1 \sim^{(n)} q'_2$.
\end{definition}

Given a prechart $(Q, E, D)$, we can equivalently see it as a pair $(Q, \beta)$, where $\beta$ is a combined transition function $Q \to \powf (\Sigma \times Q + V)$, where $\powf$ denotes a finite powerset. Such transition function $\beta$ takes each state $q \in Q$, to the set $\beta(q) = D(q) \cup E(q)$ of possible successors, that include labelled transitions and variable outputs.

In other words, precharts are coalgebras for the functor $\funL \colon \Set \to \Set$, given by $\funL = \Sigma \times (-) + V$. Bisimulations and homomorphisms of $\funL$-coalgebras are captured concretely by strong bisimulation of precharts and their homomorphisms. Because of this, we will interchangeably use terms prechart and $\funL$-coalgebra. Moreover, $\funL$ preserves weak pullbacks and hence $\sim$ is an equivalence relation that captures behavioural equivalence of $\funL$-coalgebras. For more details of coalgebraic treatment of precharts, we direct an interested reader to~\cite{Schmid:2021:Star}.
\subsection{Algebra of regular behaviours}
To define charts, Milner proposed a specification language called an \emph{algebra of regular behaviours} (ARB). The syntax of ARB is given by the following:
$$e,f \in \Expr ::= 0 \mid v \in V \mid a.e \mid e + f \mid \mu v. e$$
where $V=\{v_1, v_2, \dots\}$ and $\Sigma$ be sets of \emph{variables} and \emph{letters} respectively. Given an expression $f$ containing a variable $v$, we say that $v$ is \emph{free} in $f$, if it appears outside of the scope of the $\mu v.e$ operator or say that it is \emph{bound} otherwise. Given an expression $e \in \Expr$, we write $\fv(e) \subseteq V$ for the set of its free variables. 
\begin{definition}[{\cite{Milner:1984:Complete}}]\label{def:subset}
	Given vectors $\vec{v}$ of binders and $\vec{e}$ of expressions of the same size, we define a syntactic substitution operator $[\vec{e}/\vec{v}] \colon \Expr \to \Expr$ by the following
	\begin{gather*}
		v[\vec{e}/\vec{v}] = \begin{cases}
			\vec{e}_i & \text{if }v=\vec{v}_i\\
			v & \text{otherwise}
		\end{cases}
		\qquad
		(a.e)[\vec{e}/\vec{v}] = a.(e[\vec{e}/\vec{v}])\\
		(e + f)[\vec{e}/\vec{v}] = e[\vec{e}/\vec{v}] + f[\vec{e}/\vec{v}]\\
		(\mu w.e)[\vec{e}/\vec{v}] = \begin{cases}
			\mu w. (e[\vec{e}/\vec{v}]) & \text{if } w \text{ is not in } \vec{v} \text{ nor free in } \vec{e}\\
			\mu w. (e[z/w][\vec{e}/\vec{v}]) & \text{otherwise for some } z \text{ not in } \vec{v} \text{ nor free in } \vec{e}\\
		\end{cases}
	\end{gather*}
\end{definition}
We can now define operational semantics of ARB, by equipping its syntax with a prechart structure.
\begin{definition}[{\cite{Milner:1984:Complete}}]\label{def:operational_semantics}
	Let $(\Expr, \partial)$ be a prechart whose transition function (called \emph{derivative}) is a least one satisfying the following inference rules
	\begin{gather*}
		\inferrule{e \tr{a} e' }{a.e \tr{a} e'} \qquad \inferrule{ }{v \rhd v} \qquad \inferrule{e \tr{a} e'}{e + f \tr{a} e'} \qquad \inferrule{f \tr{a} f'}{e + f \tr{a} f'}\\
		\inferrule {e \rhd v}{e + f \rhd v} \qquad \inferrule {f \rhd v}{e + f \rhd v} \qquad \inferrule{e \rhd v\quad v \neq w}{\mu w.e \rhd v } \qquad \inferrule{e \tr{a} e'}{\mu v. e \tr{a} e'[\mu v .e / v]}
	\end{gather*}
\end{definition}
\begin{remark}[{\cite{Sewell:1995:Algebra}}]
	The last rule (that defines the transition behaviour of the $\mu$ recursion operator) can be replaced by the following:
	\begin{gather*}
		\inferrule{e[\mu v. e / v] \tr{a} e'}{\mu v. e \tr{a} e'}
	\end{gather*}
\end{remark}

\begin{remark}[{\cite[Proposition~5.4.]{Milner:1984:Complete}}]\label{rem:semantic-substitution}
	Syntactic substitution can be described operationally using the following rules
	\begin{gather*}
		\inferrule{e \rhd v\quad f\tr{a} f'}{e[f / v] \tr{a} f'}\qquad \inferrule{e \tr{a} e'}{e[f / v] \tr{a} e'[f/v]}\\ \inferrule{e \rhd w \quad w \neq v }{e[f/v] \rhd w} \qquad \inferrule{e \rhd v \quad f \rhd w}{e[f/v] \rhd w}
	\end{gather*}
\end{remark}
The syntactic prechart defined above is locally finite.
\begin{lemma}[{\cite[Proposition~5.1]{Milner:1984:Complete}}]
	For all $e \in \Expr$, $\langle e \rangle_\partial$ is finite.
\end{lemma}
We can use \Cref{c2:lem:quotient_coalgebra} and construct a following quotient $\funL$-coalgebra.
\begin{lemma}\label{lem:quotient_chart}	
	We can equip ${\Expr}/{\sim}$ with a transition function $\overline\partial$ given by
	$$
	\inferrule{e \tr{a}_\partial e'}{[e]_\sim \tr{a}_{\overline\partial} [e']_\sim} \qquad \inferrule{e \rhd_{\partial} v_i}{[e]_\sim \rhd_{\overline{\partial}} v_i}
	$$
	This map is a unique transition function on ${\Expr}/{\sim}$ that makes the quotient map $[-]_{\sim} \colon \Expr \to {\Expr}/{\sim}$ into prechart homomorphism.
\end{lemma}
We will refer to the elements of the set $\Expr/{\sim}$ as \emph{regular behaviours}.

It turns out that all operations from the syntax are compositional with respect to bisimilarity, and hence can be unambiguously lifted to the quotient prechart defined above.
\begin{lemma}[{\cite[Proposition~7]{Sewell:1995:Algebra}}]\label{lem:congruence}
	$\sim$ is a congruence on $\Expr$ with respect to all operations of the algebra of regular behaviours.
\end{lemma}
From now on, we will overload the notation and simply write $e$ for the equivalence class $[e]_{\sim}$. Using the definitions state above, we can show the following technical lemma, that we will use when constructing a semantic category for our string diagrammatic syntax.
\begin{lemma}\label{lem:subst_lemma}
	For all $e, f_1, \dots, f_m, g_1, \dots, g_m \in \Expr$ and vectors $\vec{v}=(v_{i_1}, \dots, v_{i_m}), \vec{w}=(v_{j_1}, \dots, v_{j_n})$, such that all free variables of $e$ are contained in $\vec{v}$ and all free variables of $\vec{f}$ are contained in $\vec{w}$, we have that 
	$$
	(e[\vec{f}/ \vec{v}])[\vec{g}/\vec{w}] =e [(f_1[\vec{g}/\vec{w}], \dots, f_m[\vec{g}/\vec{w}])/\vec{v}] 
	$$
\end{lemma}
\begin{proof}
	We define a relation ${R} \subseteq {\Expr \times \Expr}$, given by the following:
	\begin{align*}
		R = \text{Id} \cup \{&((e[\vec{f}/ \vec{v}])[\vec{g}/\vec{w}], e [(f_1[\vec{g}/\vec{w}], \dots, f_m[\vec{g}/\vec{w}])/\vec{v}])\\ &\mid e, f_1, \dots, f_m, g_1, \dots, g_m \in \Expr, \vec{v}=(v_{i_1}, \dots, v_{i_m}), \vec{w}=(v_{j_1}, \dots, v_{j_n}),\\ &\fv(e) \subseteq 		\vec{v},  \fv(\vec{f}) \subseteq \vec{w}\}
	\end{align*}

	
	 We claim that $R$ is a bisimulation. For pairs $(e,e) \in R$, the conditions of bisimulation are immediately satisfied. 
	
	For the remaining pairs, assume that $(e[\vec{f}/ \vec{v}])[\vec{g}/\vec{w}] \rhd u$. In such a case at least one of the following is true:
	\begin{itemize}
		\item $e[\vec{f}/ \vec{v}] \rhd u$
		\item $e[\vec{f}/\vec{v}] \rhd w_j$ for $w_j \in \vec{w}$ and $g_j \rhd u$
	\end{itemize}
	Since all free variables of $e$ are contained in $\vec{v}$ and all free variables of $\vec{f}$ are contained in $\vec{w}$, we have that all free variables of $e[\vec{f}/\vec{v}]$ are also contained in $\vec{w}$, which makes the first case impossible. Through a similar line of reasoning, we can deduce that $e \rhd v_i$ for some $v_i \in \vec{v}$ and $f_i \rhd w_j$. Since $g_j \rhd u$, we have that $f_i[\vec{g}/\vec{w}] \rhd u$. Finally, we have that $e[(f_1[\vec{g}/\vec{w}], \dots, f_n[\vec{g}/\vec{w}]), \vec{v}] \rhd u$.
	
	Assume that $(e[\vec{f}/ \vec{v}])[\vec{g}/\vec{w}] \tr{a} h$. Then, at least one of the following is true:
	\begin{itemize}
		\item $e[\vec{f}/\vec{v}] \rhd w_j$ and $g_j \tr{a} h$.
		\item $h = h'[\vec{g}/\vec{w}]$, such that $e[\vec{f}/\vec{v}] \tr{a} h'$ 
	\end{itemize}
	In the first case, through a similar line of reasoning as before, we can conclude that $e \rhd{ v_i}$ for some $v_i \in \vec{v}$ and $f_i \rhd w_j$. Hence, $f_i [\vec{g}/\vec{w}] \tr{a} h $. Finally, we can deduce that $e[(f_1[\vec{g}/\vec{w}], \dots, f_m[\vec{g}/\vec{w}])/\vec{v}] \tr{a} h$. Obviously, $(h,h) \in R$.
	
	In the second case, we have that $e[\vec{f}/\vec{v}] \tr{a} h'$. There are two subcases, that need to be considered
	\begin{itemize}
		\item $e \rhd v_i$ and $f_i \tr{a} h'$
		\item $h' = h''[\vec{f}/ \vec{w}]$ and $e \tr{a} h''$ 
	\end{itemize}
	In the first subcase, we have that $f_i[\vec{g}/ \vec{w}] \tr{a} h'[\vec{g}/\vec{w}]$ and hence $$e[(f_1[\vec{g}/ \vec{w}], \dots, f_m[\vec{g}/ \vec{w}])/\vec{v}] \tr{a} h'[\vec{g}/\vec{w}]$$ or equivalently $e[(f_1[\vec{g}/ \vec{w}], \dots, f_m[\vec{g}/ \vec{w}])/\vec{v}] \tr{a} h$. As before, of course $(h,h) \in R$. 
	
	Finally, moving on to the second subcase, we have that $e \tr{a} h''$ and hence $$e[(f_1[\vec{g}/ \vec{w}], \dots, f_m[\vec{g}/ \vec{w}])/\vec{v}] \tr{a} h''[(f_1[\vec{g}/ \vec{w}], \dots, f_m[\vec{g}/ \vec{w}])/\vec{v}]$$ Recall that $(e[\vec{f}/ \vec{v}])[\vec{g}/\vec{w}] \tr{a} h$ and $h = (h''[\vec{f}/\vec{v}])[\vec{g}/\vec{v}]$. Both of those reachable expressions are actually in the relation $R$. The remaining conditions of bisimulation, can be shown via a symmetric argument. 
\end{proof}
\subsection{Behavioural distance of precharts}
Given a pseudometric space defined on a state-space of a prechart, we can \emph{lift} it to the set of possible transitions through the following construction.
\begin{definition}[Transitions lifting]\label{def:edge}
	Let $(X,d)$ be a pseudometric space. We write $d^\uparrow$ for the pseudometric on $\Sigma \times X + V$ defined by $d^\uparrow(m, n) = \frac{1}{2} d(x,y)$ if $m = (a,x)$ and $n = (a,y)$, $d^\uparrow(m,n)=0$ if $m = n$ or $d^{\uparrow}(m,n)=1$ otherwise.
\end{definition}
This lifting admits the following:
\begin{lemma}
	$(-)^\uparrow \colon D_X \to D_{\Sigma \times X + V}$, the lifitng for $\Sigma \times (-) + V$ is contractive with respect to the metric induced by the sup norm. Namely,
	$\|d^\uparrow - d'^\uparrow \| \leq \frac{1}{2} \|d - d' \|$ 
\end{lemma}
\begin{proof}
	For the sake of simplicity, assume that $d' \sqsubseteq d$, and hence $d'^\uparrow \sqsubseteq d^\uparrow$. It suffices that we show that for all $u,w \in \Sigma \times X + V$, we have that $ d^\uparrow(u,w) - d'^\uparrow(u,w) \leq \frac{1}{2}\|d-d'\|$. Recall that in all cases, except when $u=(a,x)$ and $w=(a,y)$ for some $a \in \Sigma$ and $x,y \in X$, $d^\uparrow(u,w)=d'^\uparrow(u,w)$ and hence  $d^\uparrow(u,w)-d'^\uparrow(u,w) = 0 \leq \frac{1}{2} \|d - d'\|$. In the remaining case, we have that
	\begin{align*}
		d^\uparrow((a,x),(a,y)) - d'^\uparrow((a,x),(a,y)) &= \frac{1}{2}d(x,y) - \frac{1}{2}d'(x,y) \\
		&\leq \frac{1}{2} \|d'-d\|
	\end{align*}
	which completes the proof.
\end{proof}
Similarly, we can lift distances over $X$ to distances between elements of $\powf (X)$.
\begin{definition}[Hausdorff lifting]\label{def:hausdorff}
	Let $(X,d)$ be a pseudometric space. We can equip $\powf (X)$ with a distance function $$
	\mathcal{H}(d)(X,Y)= \max \{\sup_{x \in X} \inf_{y \in Y} d(x,y), \sup_{y \in Y} \inf_{x \in X} d(y,x) \}$$ making $(\powf(X), \mathcal{H}(d))$ into a pseudometric.
\end{definition}
Moreover, Hausdorff lifting can be equivalently characterised via the notion of \emph{relational couplings}.
\begin{remark}[{\cite[Example~5.31]{Baldan:2018:Coalgebraic}}]\label{rem:hausdorff_duality}
Let $(X, d)$ be a pseudometric space and let $A, B \in \powf(X)$. Let $\Gamma(A,B)$ denote the set of relational couplings of $A$ and $B$, namely elements $R \in \powf(A \times B)$, such that $\pi_1 (R)=A$ and $\pi_2(R)=B$. The Hausdorff distance between $A$ and $B$ can be alternatively presented as:
$$
	\mathcal{H}(d)(A,B) = \inf \left\{\sup_{(x,y) \in R} d(x,y) \mid R \in \Gamma(A,B) \right\}
$$
\end{remark}
Hausdorff lifting satisfies the following property:
\begin{lemma}[{\cite{Breugel:2012:Closure}}]
	Hausdorff lifting $\mathcal{H} \colon D_X \to D_{\powf(X)}$ is nonexpansive with respect to the metric induced by the sup norm. Namely,
	$$\|\mathcal{H}(d) - \mathcal{H}(d')\| \leq \|d - d' \|$$ 
\end{lemma}
Given a prechart $(Q, \beta)$, whose state-space is equipped with a pseudometric $d_Q$, we can define a new pseudometric $\Phi_{\beta}(d_Q)$ that calculates the distance between any pair $q_1, q_2 \in Q$ of states, by lifting $d_Q$ to the set $\powf(\Sigma \times Q + V)$ and comparing $\beta(q_1)$ with $\beta(q_2)$, namely 
	$
		\Phi_\beta(d_Q)(q_1,q_2) = \mathcal{H}\left(d_Q^\uparrow\right) (\beta(q_1), \beta(q_2))
	$. This is used to define the \emph{behavioural distance}.
	\begin{theorem}\label{thm:beh_dist}
		Let $(Q, \beta)$ be a prechart. Then, the following properties hold: \circlednum{1} $d_Q \mapsto \Phi_\beta(d_Q)$ is a monotone mapping on the lattice $D_Q$, \circlednum{2} $\Phi_\beta$ has a least fixpoint $\mathsf{bd}_\beta$, \circlednum{3} $x \sim y \implies \mathsf{bd}_\beta(x,y) = 0$ and \circlednum{4} a homomorphism $f \colon Q \to R$ between precharts $(Q, \beta)$ and $(R, \gamma)$ is an isometry between $(Q, \mathsf{bd}_\beta)$ and $(R, \mathsf{bd}_\gamma)$.
	\end{theorem}
	\begin{proof}
		$\mathcal{H}$ and $(-)^\uparrow$ are liftings for the functors $\powf$ and $\Sigma \times (-) + V$ respectively, that preserve isometries~\cite[Theorem~5.8]{Baldan:2018:Coalgebraic}. The rest follows from~\Cref{c2:lem:behavioural_distances} and \Cref{c2:lem:behavioural_distances_properties}
	\end{proof}
	The monotone map used to define the behavioural distance satisfies the following:
\begin{lemma}
	$\Phi_\beta \colon D_X \to D_X$ is contractive with respect to the metric induced by the sup norm, namely
	$$
	\|\Phi_\beta(d)-\Phi_\beta(d')\| \leq \frac{1}{2}\|d-d'\|
	$$
\end{lemma}
\begin{proof}
	For the sake of simplicity, assume that $d' \sqsubseteq d$ and hence $\Phi_\beta(d') \sqsubseteq \Phi_{\beta}(d)$. It suffices to show that for all $x,y \in X$, we have that $\mathcal{H}(d^\uparrow)(\beta(x), \beta(y)) -  \mathcal{H}(d'^\uparrow)(\beta(x), \beta(y))  \leq \frac{1}{2}\|d-d'\|$. We can combine the previous results and for arbitrary $x,y \in X$ obtain the following
	\begin{align*}
		\mathcal{H}(d^\uparrow)(\beta(x), \beta(y)) -  \mathcal{H}(d'^\uparrow)(\beta(x), \beta(y)) & \leq \|\mathcal{H}(d^\uparrow) - \mathcal{H}(d'^\uparrow)\| \\
		&\leq \|d^\uparrow - d'^\uparrow\|\\
		& \leq \frac{1}{2} \|d - d'\|
	\end{align*}
\end{proof}
As a consequence, we have the following corollary:
\begin{corollary}
	$\Phi_\beta$ has a unique fixpoint.
\end{corollary}
Additionally, through an identical argument to \Cref{c2:lem:cocontinuous}, we can show the following:
\begin{lemma}\label{lem:cocontinuous}
	For a finite prechart $(X,\beta)$, $\Phi_\beta$ is cocontinuous.
\end{lemma}
Since $\Phi_\beta$ has a unique fixpoint, we can use~\Cref{c2:thm:kleene} to calculate behavioural distance of states of finite precharts.
\begin{lemma}\label{lem:finite_dist}
	Let $(Q, \beta)$ be a finite prechart. The behavioural distance between any pair $q_1, q_2 \in Q$ of states can be calculated by $\mathsf{bd}_\beta(q_1,q_2) = \inf_{p \in \N} \left \{ \Phi^{(p)}_\beta(q_1, q_2) \right\}$, where $\Phi^{(0)}_\beta$ is a discrete pseudometric and for any $p \in \N$, we define $\Phi^{(p+1)}_\beta = \Phi_\beta \left( \Phi^{(p)}_\beta\right)$.
\end{lemma}
The characterisation described above can be extended to any locally finite prechart.
\begin{corollary}\label{cor:kleene_locally_finite}
	For any locally finite prechart $(X,\beta)$, the distance between $x, y \in X$, can be calculated by:
	\[
		\mathsf{bd}_\beta(x,y) = \inf_{i \in \N} \left(\Phi_\beta^{(i)}(x,y)\right)
	\]
\end{corollary}
\begin{proof}
Let $\beta'$ denote $\beta$ restricted to $\langle x , y \rangle_\beta$.
	Since $(X, \beta)$ is locally finite, then its subprechart $(\langle x , y \rangle_\beta, \beta')$ is finite. Since homomorphisms are ismometries, calculating distance between $x$ and $y$ in $(X, \beta)$ is the same as calculating it in $(\langle x , y \rangle_\beta, \beta')$. Because of \Cref{lem:cocontinuous}, $\Phi_\beta$ is cocontinuous (when restricted to $\langle x , y \rangle_{\beta'}$) and hence we can employ \Cref{c2:thm:kleene}. Since the infima in the lattice of pseudometrics can be calculated pointwise (\Cref{c2:lem:chain_pointwise_inf}), we have that
	$$
	\mathsf{bd}_\beta(x,y) = \mathsf{bd}_{\beta'}(x,y) = \inf_{i \in I} \left( \Phi^{(i)}_{\beta'}(x,y)\right)
	$$
	Since $\beta'$ is a restriction of $\beta$ to ${\langle x , y \rangle_\beta}$ and each $\Phi^{(i)}_{\beta'}$ makes only use of the states in $\langle x , y \rangle_\beta$, we can rewrite the above as
	$$
	\mathsf{bd}_\beta(x,y) = \inf_{i \in \N} \left(\Phi_\beta^{(i)}(x,y)\right)
	$$
	as desired.
\end{proof} 
\begin{lemma}\label{lem:distance_power_of_two}
	Let $(X,\beta)$ be a locally finite prechart. For all $x,y \in X$, $i \in \N$, either $\Phi_\beta^{(i)}=0$ or there exists $k \in \N$, such that $\Phi_{\beta}^{(i)}(x,y)=2^{-k}$
\end{lemma}
\begin{proof}
	Pick arbitrary $x,y \in X$.
	By induction on $i$. If $i = 0$, then $\Phi^{(0)}_\beta(x,y)=1=2^{0}$, if $x \neq 0$ or $\Phi^{(0)}_\beta(x,y)=0$, otherwise. 
	If $i = j + 1$, then we have that:
	$$
	\Phi^{j + 1}_\beta(x,y) = \max \{\sup_{u \in \beta(x)} \inf_{w \in \beta(y)} {\Phi^{(j)}_\beta}^\uparrow(u,w), \sup_{w \in \beta(y)} \inf_{u \in \beta(x)}{\Phi^{(j)}_\beta}^\uparrow(w,u)\}.
	$$
	
	Recall that for $u = (a,x')$ and $w=(a,y')$, we have that ${\Phi^{(j)}_\beta}^\uparrow(u,w)=\frac{1}{2} {\Phi^{(j)}_\beta}(x',y')$. By induction hypothesis, we have that ${\Phi^{(j)}_\beta}(x',y')=0$ and hence ${\Phi^{(j)}_\beta}^\uparrow(u,w)=0$ or there exists a $k \in N$, such that ${\Phi^{(j)}_\beta}(x',y')=2^{-k}$ and hence ${\Phi^{(j)}_\beta}^\uparrow(u,w)=2^{-(k+1)}$. Since the infima range over finite sets, their values are either $1=2^0$ if the sets are empty or are one of the values from the set, which we have shown to be of the desired form. Similarly, suprema range over finite sets and are either $0$ for empty sets or are on of the values from the set, which are in the desired form. Taking the maximum of values in the desired form, still results in a value in the desired form.
\end{proof}

\subsection{Conway Theories}\label{c3:subsec:conway}
Let $\cat{C}$ be a category, whose objects are natural numbers and $0$ is the initial object. We will write $0_n$ for the unique map $0_n \colon 0 \to n$. Additionally, we assume that $\cat{C}$ is equipped with all finite coproducts, where binary coproduct is given by addition, i.e. $n \oplus m := n + m$. Given $f \colon k \to m$ and $g \colon l \to m$, we will write $\lc f,g \rc \colon k +l \to m$ for the mediating map from the universal property of the coproduct that makes the following diagram commute:
\begin{equation}\label{eqn:mediating}
\begin{tikzcd}
	k && l \\
	& {k+l} \\
	\\
	& m
	\arrow["\inl_{k + l}", from=1-1, to=2-2]
	\arrow["f"', from=1-1, to=4-2]
	\arrow["\inr_{k + l}"', from=1-3, to=2-2]
	\arrow["g", from=1-3, to=4-2]
	\arrow["{\lc f , g \rc}"{description}, dashed, from=2-2, to=4-2]
\end{tikzcd}
\end{equation}
For every $n \in N$, we can define a codiagonal $\nabla_{n} \colon n + n \to n$, given by $\nabla_n := \lc\id_n,\id_n\rc$.

Given $f \colon k \to l$ and $g \colon m \to n$, we can define their \emph{separated sum} $f \oplus g \colon k + m \to l + n $, given by the unique mediating arrow in the following diagram
% https://q.uiver.app/#q=WzAsNixbMCwwLCJrIl0sWzIsMCwiayArIG4iXSxbMCwyLCJsIl0sWzQsMCwibSJdLFs0LDIsIm4iXSxbMiwyLCJsICsgbiJdLFswLDEsIlxcaW5sIl0sWzMsMSwiXFxpbnIiLDJdLFsyLDUsIlxcaW5sIiwyXSxbNCw1LCJcXGluciJdLFsxLDUsImYgXFxvcGx1cyBnIiwxLHsic3R5bGUiOnsiYm9keSI6eyJuYW1lIjoiZGFzaGVkIn19fV0sWzAsMiwiZiIsMl0sWzMsNCwiZyJdXQ==
% https://q.uiver.app/#q=WzAsNixbMCwwLCJrIl0sWzIsMCwiayArIG0iXSxbMCwyLCJsIl0sWzQsMCwibSJdLFs0LDIsIm4iXSxbMiwyLCJsICsgbiJdLFswLDEsIlxcaW5sIl0sWzMsMSwiXFxpbnIiLDJdLFsyLDUsIlxcaW5sIiwyXSxbNCw1LCJcXGluciJdLFsxLDUsImYgXFxvcGx1cyBnIiwxLHsic3R5bGUiOnsiYm9keSI6eyJuYW1lIjoiZGFzaGVkIn19fV0sWzAsMiwiZiIsMl0sWzMsNCwiZyJdXQ==
\begin{equation}\begin{tikzcd}\label{eqn:monoidal}
	k && {k + m} && m \\
	\\
	l && {l + n} && n
	\arrow["\inl_{k,m}", from=1-1, to=1-3]
	\arrow["f"', from=1-1, to=3-1]
	\arrow["{f \oplus g}"{description}, dashed, from=1-3, to=3-3]
	\arrow["\inr_{k,m}"', from=1-5, to=1-3]
	\arrow["g", from=1-5, to=3-5]
	\arrow["\inl_{l,n}"', from=3-1, to=3-3]
	\arrow["\inr_{l,n}", from=3-5, to=3-3]
\end{tikzcd}\end{equation}
\begin{remark}
	Because of the universal properties of coproduct and initial object, the following identities hold:
	\begin{enumerate}
		\item $\lc f , \lc g , h \rc \rc = \lc \lc f , g \rc, h \rc$
		\item $\lc 0_n, f \rc = f = \lc f , 0_n \rc$
		\item $\lc f , g \rc ; h = \lc f ; h , g ; h \rc$
		\item $(f \oplus g) \oplus h = f \oplus (g \oplus h )$
		\item $0_n \oplus f = f = f \oplus 0_n$
		\item $(f \oplus g) ; (h \oplus i) = f ; h \oplus g;i$
	\end{enumerate}
\end{remark}
Because of the above remark, we can unambiguously define an $n$-ary version of $\lc -, -\rc$ and hence we can view every map $f \colon m \to n$, as $f = \lc f_1, \dots, f_m\rc$.

Observe that under the assumptions listed above $\cat{C}$ is equipped with all finite coproducts (since it has an initial object and binary coproducts), and hence $(\cat{C}, \oplus , 0 )$ is a cocartesian strict symmetric monoidal category. 

 We call $\cat{C}$ a \emph{preiteration theory} if for every morphism $f \colon n \to p + n$, there exists a morphism $f^\dagger_{n,p} \colon n \to p$ called \emph{dagger}. We will often omit the subscripts and write $f^\dagger$, when $n$ and $m$ are clear from the context. Note that the definition does not impose any conditions on the dagger. However, for $f \colon 0 \to p$, when always we have that $f_{0,p}^\dagger = 0_p$.
 
 
 
\begin{definition}[{\cite[Definition~3.1]{Esik:1999:Group}}]\label{def:conway_theory}
	A Conway theory is a preiteration theory, in which the following conditions are satisfied:
	\begin{itemize}
		\item \textbf{(Scalar parameter identity)} $$(f ; (g \oplus \id_1))^\dagger = f^\dagger ; g$$ for all $f \colon 1 \to p + 1$, $g \colon p \to q$.
		\item \textbf{(Scalar composition identity)} 
		$$
		(f ; \lc \id_p\oplus0_1, g \rc)^\dagger = f ; \lc \id_p,  (g ;  \lc \id_p \oplus 0_1 , f\rc)^\dagger\rc
		$$ for all $f,g \colon 1 \to p + 1$.
		\item \textbf{(Scalar double dagger identity)} 
		$$f^{\dagger\dagger} = (f ; (\id_p \oplus \nabla_1))^\dagger$$
		for all $f \colon 1 \to p + 2$.
		\item \textbf{(Scalar pairing identity)}
		$$\lc f ,g \rc^\dagger = \lc f^\dagger ; \lc \id_p, h^\dagger \rc, h^\dagger \rc$$ for all $f \colon n \to p + 1 + n $, $g \colon 1 \to p + 1 + n$ where 
		$$
		h = g ; \lc \id_{p + 1}, f^\dagger\rc \colon 1 \to p + 1
		$$
	\end{itemize}
\end{definition}
\begin{remark}[{\cite[Remark~3.2]{Esik:1999:Group}}]\label{rem:defining_dagger}
	Note that in order to define a Conway theory it suffices to define $f^\dagger \colon 1 \to p$ for all $f \colon 1 \to p + 1$ that satisfies first three axioms of \Cref{def:conway_theory} and use \textbf{scalar pairing identity} to inductively define $(-)^\dagger$.
\end{remark}

\subsection{Trace-fixpoint correspondence}
It turns out, that having a category $\cat{C}$ with finite coproducts and equipped with a dagger operator satisfying the axioms of Conway theories is synonymous with $\cat{C}$ being traced symmetric monoidal category. This is captured by the following theorem that was independently proved by Hasegawa~\cite{Hasegawa:1997:Models} and Haghverdi~\cite{Haghverdi:2000:Categorical}. The formulation of Hasegawa is phrased dually via the setting of products and cartesian categories.
\begin{theorem}[{\cite[Proposition~3.1.9]{Haghverdi:2000:Categorical}}]\label{thm:trace}
For any category with finite coproducts, to give a Conway operator is to give a trace (where finite coproducts are taken as the monoidal structure). 
\end{theorem}
That bijective correspondence is concretely given by the following:
\begin{gather*}
	\inferrule{f \colon  n \to p + n}{f^\dagger = \Tr^n_{n,p} (\nabla_n ; f) \colon n \to p} 
	\qquad
	\inferrule{g \colon p + n \to q + n}{\Tr^{n}_{p,q}(g) = \inl_{p,n} ; (g; \lc\id_q, \inr_{q +p,n} \rc)^\dagger  \colon p \to q}
\end{gather*}	

\subsection{Int construction}
Given a traced symmetric monoidal category $(\cat{C}, \otimes, I)$, we can construct a compact closed category $\Int{\cat{C}}$. The objects of $\Int{\cat{C}}$ are the pairs $(A^+,A^-)$ of objects of $\cat{C}$. Morphisms $f$ from $(A^+, A^-)$ to $(B^+, B^-)$ are the morphisms $f \colon A^+ \otimes B^- \to A^- \otimes B^+$ of $\cat{C}$. The identity of any object $(A^+, A^-)$ is given by the symmetry of $\cat{C}$, namely $\id_{(A^ +, A^-)} = \sigma_{A^+, A^-}$. The composition $f ; g \colon (A^+, A^-) \to (C^+, C^-)$ of morphisms $f \colon (A^+, A^-) \to (B^{+}, B^-)$ and $g \colon (B^+, B^-) \to (C^+,C^-)$ is defined as
	$
		\Tr^{B^- \otimes B^+}_{A^+ \otimes C^-, A^- \otimes C^+} (\alpha ; (f \otimes g) ; \beta)
	$, where \begin{gather*}\alpha = (\id_{A^+} \otimes \sigma_{C^-, B^-} \otimes \id_{B^+});(\id_{A^+} \otimes \id_{B^-} \otimes \sigma_{C^-, B^+})\\\beta = (\id_{A^-} \otimes \id_{B^+} \otimes \sigma_{B^-, C^+}); (\id_{A^-} \otimes \sigma_{B^+, C^+} \otimes \id_{B^-});(\id_{A^-} \otimes \id_{C^+} \otimes \sigma_{B^+, B^-})\end{gather*}
	
	$\Int{\cat{C}}$ is equipped with the monoidal structure. The tensor product of $(A^+, A^-)$ and $(B^+, B^-)$ is given by taking the tensor product of $\cat{C}$ pointwise, namely $(A^+ \otimes B^+, A^- \otimes B^-)$. The unit of that monoidal product is given by $(I, I)$, where $I$ is the unit of the monoidal product on $\cat{C}$. The tensor product $f \otimes g \colon (A^+ \otimes C^+, B^- \otimes D^-) \to (A^- \otimes C^-, B^+ \otimes D^+)$ of $f \colon (A^+, A^-) \to (B^+, B^-)$ and $g \colon (C^+, C^-) \to (C^+, C^-) \to (D^+, D^-)$ is given by the following:
	$$
	f \otimes g = (\id_{A^+} \otimes \sigma_{C^+, B^-} \otimes \id_{D^-});(f \otimes g);(\id_{A^-} \otimes \sigma_{B^+, C^-} \otimes \id_{D^+})
	$$
	
	The dual $(A^+, A^-)^{\ast}$ of $(A^+, A^-)$ is given by exchanging the components, that is by $(A^-, A^+)$. Then, the unit $\eta_{(A^+, A^-)} \colon (I,I) \to (A^+, A^-) \otimes (A^+, A^-)^{\ast}$ is a morphism $\sigma_{A^-, A^+} \colon A^- \otimes A^+ \to A^+ \otimes A^-$. The counit $\epsilon_{(A^+, A^-)} \colon (A^+, A^-)^\ast \otimes (A^+, A^-) \to (I,I)$ can be similarly given by $\sigma_{A^-, A^+} \colon A^- \otimes A^+ \to A^+ \otimes A^-$ in $\cat{C}$.
	
	$\Int{\cat{C}}$ is equipped with a canonical trace, which takes a morphism $$f \colon (A^+, A^-) \otimes (U^+, U^-) \to (B^+, B^-) \otimes (U^+, U^-)$$ to the map given by the following:
	\[
	\left(\id_{(A^+, A-)} \otimes \eta_{(U^+, U^-)}\right) ;\left(f \otimes \id_{{(U^+, U^-)}^\ast}\right); \left(\id_{(B^+, B^-)} \otimes \epsilon_{(U^+, U^-)}  \right)
	\]
	
	\section{Monoidal Syntax}\label{c3:sec:monoidal}
We adopt the diagrammatic syntax for NFA that has appeared in a number of previous papers~\cite{piedeleu2023finite,antoinecsl2025}. 
We refer the reader to Selinger's classic survey~\cite{Selinger_2010}, or to Piedeleu and Zanasi's recent text for a more gentle introduction to the language of string diagrams~\cite{piedeleu2023introduction}.

This syntax is formalised as a product and permutation category, or prop, a structure which generalises algebraic theories. Formally, a \emph{prop} is a strict symmetric monoidal category (SMC) whose objects are words over a set of generators and whose monoidal product $\proptimes$ is given by concatenation. 
More specifically, our syntax is the free prop $\freeP{\Signature}$ over the signature $\Signature = (\Obj,\Morph)$, given by a set $\Obj$ of generating objects and a set $\Morph$  of generating morphisms $g\from v\to w$, with $v,w\in \Obj^*$ (we use $\epsilon$ to denote the empty word). Morphisms of  $\freeP{\Signature}$ can be combined in two different ways, using the composition operation $(-);(-)\from \freeP{\Signature}(u,v)\times \freeP{\Signature}(v,w)\to \freeP{\Signature}(u,w)$ or the monoidal product $(-)\proptimes(-)\from \freeP{\Signature}(v_1,w_1)\times \freeP{\Signature}(v_2,w_2)\to \freeP{\Signature}(v_1 v_2,w_1 w_2)$. We also have distinguished constants: identities $\id_w\from w\to w$, which are the unit for composition, and symmetries $\sigma^v_w\from vw\to wv$, to reorder the letters of a given object. In summary, morphisms of $\freeP{\Signature}$ can be described as terms of the $(\Obj^*,\Obj^*)$-sorted syntax generated from the constants $\Morph + \{\id_w : w\in \Obj^*\}+ \{\sigma^v_w : v,w\in \Obj^*\}$ using the operations $;$ and $\proptimes$, \emph{quotiented} by the axioms of SMCs. However, the terms of this syntax are very cumbersome to work with. 

We adopt a more convenient way to represent morphisms of $\freeP{\Signature}$, using the graphical notation of \emph{string diagrams}. In this view, a morphism $f\from v\to w$ of $\freeP{\Signature}$ is depicted as a $f$-labelled box with a $v$-labelled on the left and a $w$-labelled wire on the right. The operations of composition and monoidal product are represented by connecting two boxes horizontally and juxtaposing two boxes vertically, respectively:
\begin{equation*}\label{eq:composition-monoidal-product}
\tikzfig{comp-sequential-fg}\qquad \quad \tikzfig{comp-parallel-fg}
\end{equation*}
Wires $\tikzfig{id-w}$ represent identities, the wire crossing $\tikzfig{sym-vxw}$ represents the symmetry $\sigma^v_w$, and the empty diagram $\idzero$ the identity $\id_\epsilon\from \epsilon\to \epsilon$.
\begin{definition}\label{c3:def:syntax}
	 We call $\Syn$ the free prop over the signature given by
	\begin{itemize}
		\item two generating objects $\objl$ ("left") and $\objr$ ("right"), with their identity morphisms depicted respectively as $\arrowleft$ and $\arrowright$;
 		\item generating morphisms $
 		\tikzfig{lr-copy}\quad\tikzfig{lr-delete}\quad\tikzfig{lr-merge}\quad\tikzfig{lr-generate} \quad\tikzfig{cap-down} \quad\tikzfig{cup-down}\quad \scalar{a} \quad (a\in \Sigma)$.
 	\end{itemize}
 \end{definition}
Morphisms of $\Syn$ are thus vertical and horizontal composites of the generators above, potentially including wire crossings and identity wires, \emph{up to} the laws of symmetric monoidal categories, listed below:

\begin{equation*}
\begin{gathered}
{\tikzfig{smc/sequential-associativity} \SMCeq \tikzfig{smc/sequential-associativity-1}} \qquad {\tikzfig{smc/parallel-associativity} \SMCeq \tikzfig{smc/parallel-associativity-1}}\\  
{\tikzfig{smc/interchange-law}\SMCeq\tikzfig{smc/interchange-law-1} }
 \qquad
{\tikzfig{smc/unit-right} \SMCeq \diagbox{c}{}{} \SMCeq \tikzfig{smc/unit-left}}
\\
{ \tikzfig{smc/parallel-unit-above} \SMCeq \diagbox{c}{}{} \SMCeq  \tikzfig{smc/parallel-unit-below}}
\qquad
{\tikzfig{smc/sym-natural} \SMCeq \tikzfig{smc/sym-natural-1}}
\\		
{\tikzfig{smc/sym-iso} \SMCeq \tikzfig{id2}}
\end{gathered}
\end{equation*}


 The direction of the arrows on the wires denotes the corresponding object: for example, $\tikzfig{lr-copy}$ represents an operation of type $\objr\to \objr\objr$, while $\tikzfig{cap-down}$ has type $\objl\objr\to \epsilon$. Note that, when we have $n$ parallel wires of the same type, say $\objr$, we depict them as a single directed wire labelled by a natural number label, as $\idright^{\!\!\!\!\!\! n}$. We call \emph{inputs} the incoming wires of a diagram, and \emph{outputs} its outgoing wires; formally, the inputs (resp. outputs) of $f\from v\to w$ are the set of positions of the word $v$ which are $\objr$ (resp. $\objl$) and the position of $w$ which are $\objl$ (resp. $\objr$).
	
	
	\section{Monoidal semantics}
	
	In order to interpret the string diagrams described in \Cref{c3:sec:monoidal}, we construct an appropriate semantic universe out of regular behaviours. As much as the technical development makes use of category theory, we will keep the description of the formalism high-level. We will write $V_n$ for the set $V_n = \{v_1, \dots, v_n\}\subseteq V$ and $\Expr/{\equiv}(n)$ for the set of all regular behaviours whose live variables are contained in the set $V_n$. For any $m,n \in \N$, we will write $\RegBeh(m,n)$ for the set of $m$-tuples of elements of $\Expr/{\equiv}(n)$. 
	
	For every $n \in N$, we define an identity map $\id_n \in \RegBeh(n,n)$ as  $\id_n=\vec{v}_n=(v_1, \dots, v_n)$. When $n$ is clear from the context, we will abuse the notation and simply write $\vec{v}$ instead.

	Given $f \in \RegBeh(m,n)$ and $g \in \RegBeh(n,p)$, we can define their sequential composition $f;g \in \RegBeh(m,p)$ to be given by $(f_1[\vec{g} / \vec{v}], \dots, f_m[\vec{g}/ \vec{v}])$, where $\vec{v}=(v_1, \dots, v_n)$. It turns out that sequential composition is associative, with identity being a neutral element when composed both on the left and right.
 \begin{restatable}{lemma}{regbehcategory}\label{lem:comp_associative}
	Let $f \colon m \to n$, $g \colon n \to p$, $h \colon p \to q$. We have that: 
	\begin{enumerate}
		\item $(f;g);h = f;(g;h)$
		\item $\id_m ; f = f$
		\item $f ; \id_n = f$
	\end{enumerate}	
\end{restatable}
\begin{proof}
	We respectively prove each of the properties.
	\begin{enumerate}
		\item \begin{align*}
			(f ; g) ; h &= (f_1[\vec{g}/\vec{v}], \dots, f_m[\vec{g}/\vec{v}]);h \\
			&= \left((f_1[\vec{g}/\vec{v}])[\vec{h}, \vec{v}], \dots, (f_m[\vec{g}/\vec{v}])[\vec{h}, \vec{v}]\right)\\
			&= \left(f_1[(g_1[\vec{h}/\vec{v}], \dots, g_n[\vec{h}/\vec{v}])/\vec{v}],\dots, f_m[(g_1[\vec{h}/\vec{v}], \dots, g_n[\vec{h}/\vec{v}])/\vec{v}]\right) \tag{\Cref{lem:subst_lemma}} \\
			&= \left(f_1 [\vec{(g;h)}/\vec{v}], \dots f_m [\vec{(g;h)}/\vec{v}] \right) \\
			&= f ; (g;h)
		\end{align*}
		\item $id_m ; f = (v_1[\vec{f}/\vec{v}], \dots, v_m[\vec{f}/\vec{v}])=(f_1, \dots, f_m) = f$
		\item $f ; id_n = (f_1[\vec{v}/\vec{v}], \dots, f_m[\vec{v}/\vec{v}]]) = (f_1, \dots, f_m) = f$
	\end{enumerate}
\end{proof}

Because of the above, we can define a category $\RegBeh$, whose objects are natural numbers and morphisms $f \colon m \to n$ are elements $f \in \RegBeh(m,n)$. 

 For every $n \in \N$, there is a unique element $0_n \in \RegBeh(0,n)$ given by the empty tuple.
 \begin{lemma}
	$0$ is the initial object of $\RegBeh$.
\end{lemma}
\begin{proof}
	For any $n \in \N$, the unique universal arrow is given by $0_n$, which immediately completes the proof.
\end{proof}
 Given $f \in \RegBeh(k,m)$ and $g \in \RegBeh(l,m)$, we define their pairing $\langle f , g \rangle \in \RegBeh (k + l,m)$, by setting $\lc f , g \rc = (f_1, \dots, f_k, g_1, \dots g_l)$. $\RegBeh$ can be equipped with binary coproducts, which is defined on objects as addition and the mediating map is given by pairing.
\begin{lemma}
	$\RegBeh$ has binary coproducts. In particular, given $k, l \in \N$, the inclusions $\inl_{k,l} \colon k \to k +l$ and $\inr_{k,l} \colon l \to k + l$ are given by $\inl_{k,l} = (v_1, \dots, v_k)$ and $\inr_{k,l} = (v_{k+1}, \dots, v_{k+l})$ respectively, while the mediating map is given by pairing. 
% https://q.uiver.app/#q=WzAsNCxbMCwwLCJrIl0sWzIsMCwibCJdLFsxLDEsImsrbCJdLFsxLDMsIm0iXSxbMCwyLCJcXGlubCJdLFsxLDIsIlxcaW5yIiwyXSxbMCwzLCJmIiwyXSxbMSwzLCJnIl0sWzIsMywiXFxsYW5nbGUgZiAsIGcgXFxyYW5nbGUiLDEseyJzdHlsZSI6eyJib2R5Ijp7Im5hbWUiOiJkYXNoZWQifX19XV0=
\end{lemma}
\begin{proof}
	Let $f \colon k \to m$ and $g \colon l \to m$. We can safely assume that $f = (f_1, \dots, f_k)$ and $g = (g_1, \dots, g_l)$. Recall that $\inl_{k,l} = (v_1, \dots, v_k)$ and $\inr_{k,l} = (v_{k+1}, \dots, v_{k+l})$. For the existence proof, define $\lc f , g \rc \colon k + l \to m$ as a $k + l$-tuple $(f_1, \dots, f_k, g_1, \dots, g_l)$. We show that that the coproduct diagram from \Cref{eqn:mediating} commutes. We start from the left triangular subdiagram.
%	\[\begin{tikzcd}
%	k && l \\
%	& {k+l} \\
%	\\
%	& m
%	\arrow["\inl_{k + l}", from=1-1, to=2-2]
%	\arrow["f"', from=1-1, to=4-2]
%	\arrow["\inr_{k + l}"', from=1-3, to=2-2]
%	\arrow["g", from=1-3, to=4-2]
%	\arrow["{\lc f , g \rc}"{description}, dashed, from=2-2, to=4-2]
%\end{tikzcd}\]
	\begin{align*}
		\inl_{k,l} ; \lc f , g\rc &= (v_1, \dots, v_k) ; (f_1, \dots f_k, g_1, \dots, g_l) \\
		&= (f_1, \dots, f_k) \\
		&= f
	\end{align*}
	Similarly, for the right subdiagram, we have that:
	\begin{align*}
		\inr_{k,l} ; \lc f , g\rc &= (v_{k +1}, \dots, v_{k + l}) ; (f_1, \dots f_k, g_1, \dots, g_l) \\
		&= (g_1, \dots, g_l) \\
		&= g
	\end{align*}
	For the uniqueness proof, assume that there exists a map $h \colon k + l \to m$, which makes the coproduct diagram commute. We can safely assume that $h = (h_1, \dots, h_{k + l})$. 
	Since $\inl_{k,l} ; h = f$ and $\inr_{k,l} ; h = g$, we have that:
	\begin{align*}
		(f_1, \dots, f_k) &= f\\ 
		&= \inl_{k,l} ; h\\
		&= (v_1, \dots, v_k) ; (h_1, \dots, h_{k + l}) \\
		&= (h_1, \dots, h_k)
	\end{align*}
	Similarly, we have that:
	\begin{align*}
		(g_1, \dots, g_l) &= f\\ 
		&= \inr_{k,l} ; h\\
		&= (v_{k+1}, \dots, v_{}) ; (h_1, \dots, h_{k + l}) \\
		&= (h_{k+1}, \dots, h_{k + l})
	\end{align*}
	Hence, $h = (f_1, \dots, f_k, g_1, \dots, g_l) = \lc f , g \rc$ as desired. 
\end{proof}
Given $f \in \RegBeh(k,l)$ and $g \in \RegBeh(m,n)$, we can define their parallel composition $f \oplus g \in \RegBeh(k + m, l + n)$, by setting $$f \oplus g = (f_1, \dots, f_k, g_1[(v_{l + 1},\dots, v_{l + n})/\vec{v}], \dots, g_m[(v_{l + 1},\dots, v_{l + n})/\vec{v}])$$ One can easily verify that this corresponds to the map given by \Cref{eqn:monoidal}. As a consequence of the universal properties established above, we have that $(\RegBeh, \oplus , 0 )$ is a co-cartesian strict symmetric monoidal category. 
\subsection{$\RegBeh$ as a Conway theory}
We move on to showing that $\RegBeh$ is in fact a Conway theory. For any $f \in \RegBeh(1, p + 1)$, we can define $f^\dagger \in \RegBeh(1,p)$ to be given by $f^\dagger = \mu v_{p+1}.f$.
The dagger defined above satisfies \textbf{scalar parameter identity}.
\begin{lemma}\label{conway1}
	Let $f \colon 1 \to 1 + p$, $g \colon p \to g$ be morphisms in $\RegBeh$. Then,
	$$
	(f ; (g\oplus \id_1))^\dagger = f^\dagger ; g
	$$
\end{lemma}
\begin{proof}
\begin{align*}
(f ; (g \oplus \id_1))^\dagger & = (f[\vec{g}/\vec{v}])^\dagger
\\
& = \mu v_{p+1}.\big(f[\vec{g}/\vec{v}]\big) \tag{Def. of $\dagger$}
\\
& = (\mu v_{p+1}.f)[\vec{g}/\vec{v}] 
\\
& = f^\dagger[\vec{g}/\vec{v}] 
\\
& = f^\dagger ; g
\end{align*}
\end{proof}
To show the remaining identities, we first recall the following lemma.
 \begin{lemma}[{\cite[Theorem~2]{Sewell:1995:Algebra}}]
\label{lem:recursion-substitution}
Terms of Milner's ARB (modulo bisimilarity) satisfy the following rules:
	\begin{enumerate}
		\item $\mu v_z. \left(e [(v_z, v_z) / (v_j, v_k)]\right) = \mu v_j. \mu v_k. e$ for any $v_z$ not free in $e$
		\item $ \mu v_j.\left(e[f/v_j]\right) = e[\mu v_x. \left(f[e/v_j]\right)/v_j]$
	\end{enumerate}
\end{lemma}
We can now establish the \textbf{scalar composition} indentity.
\begin{lemma}\label{conway2}
Let $f,g \colon 1 \to 1 + p$ be morphisms of $\RegBeh$. Then,
		$$
		(f ; \langle  \id_p \oplus 0_1 , g \rangle)^\dagger = f ; \langle \id_p, (g ;  \langle  \id_p \oplus 0_1 , f\rangle)^\dagger\rangle
		$$ 
\end{lemma}
\begin{proof}
\begin{align*}
(f ; \langle  \id_p \oplus 0_1 , g \rangle)^\dagger & = \big(f[v_1,\dots,v_p, g/\vec{v}]\big)^\dagger	
\\
& = \mu v_{p+1}.\big(f[v_1,\dots,v_p, g/\vec{v}]\big)
\\
& = \mu v_{p+1}.\big(f[g/v_{p+1}]\big)
\\
& = f[\mu v_{p+1}. \left(g[f/v_{p+1}]\right)/v_{p+1}] \tag{\Cref{lem:recursion-substitution}~2.}
\\
& = f[\mu v_{p+1}. \left(g[v_1,\dots,v_n,f/v_1,\dots,v_n,v_{p+1}]\right)/v_{p+1}]\\
& = f[\mu v_{p+1}. (g ;  \langle  \id_p \oplus 0_1 , f\rangle)/v_{p+1}]
\\
& = f[(g ;  \langle  \id_p \oplus 0_1 , f\rangle)^\dagger/v_{p+1}]
\\
& = f ; \langle \id_p, (g ;  \langle  \id_p \oplus 0_1 , f\rangle)^\dagger\rangle
\end{align*}
\end{proof}
Similarly, we can show that dagger on $\RegBeh$ satisfies \textbf{scalar double dagger identity}. 
\begin{lemma}\label{conway3}
Let $f \colon 1 \to 2 + p$ be a morphism of $\RegBeh$. Then,
		$$
		f^{\dagger\dagger} = (f ; (\id_p\oplus \nabla_1))^\dagger
		$$
\end{lemma}
\begin{proof}
\begin{align*}
f^{\dagger\dagger} &= \mu v_{p+1}.(\mu v_{p+2}.f)
\\
& = \mu v_{p+1}. \left(f [v_{p+1}, v_{p+1} / v_{p+1}, v_{p+2}]\right) \tag{\Cref{lem:recursion-substitution}~ 1.}
\\
& = \mu v_{p+1}. \left(f;(\id_p\oplus \nabla_1)\right)
\\
& = (f ; (\id_p\oplus \nabla_1))^\dagger
\end{align*}
\end{proof}
Hence, we obtain the desired result.
\begin{restatable}{lemma}{regbehconway}\label{lem:regbehconway}
$\RegBeh$ is a Conway Theory.	
\end{restatable}
\begin{proof}
	Follows from \Cref{conway1}, \Cref{conway2} and \Cref{conway3}.
\end{proof}
We can now combine all the intermediate results into the following statement.
\begin{theorem}
	The category $\RegBeh$ has the following properties:
	\begin{itemize}
		\item $\RegBeh$ has all finite coproducts. 
		\item $(\RegBeh, \oplus, 0)$ is a (co-Cartesian) strict symmetric monoidal category.
		\item $\RegBeh$ equipped with a dagger is a Conway theory~\cite{Esik:1999:Group}.
		\item Each morphism $g \colon p + n \to q + n$ has a trace $\Tr^{n}_{p,q}(g) \colon p \to q$ defined in terms of dagger. This equipment makes $\RegBeh$ into a traced monoidal category~\cite{Joyal:1996:Traced}. 
	\end{itemize}
\end{theorem} 
 